\documentclass{beamer}
\usepackage{beamerthemesplit}
\usepackage[T1]{fontenc}
\usepackage[austrian]{babel}
\usepackage[utf8]{inputenc}
\usepackage{amsmath}
\usepackage{listings}

\u

\title{Debugging für parallele Programme}
\author{Bernhard Fritz}
\date{\today}
\logo{\includegraphics[height=1.5cm]{../Bilder/Uni_Logo_4C.pdf}}

\makeatletter
       \newcount\c@p
       \newcount\c@m
              \newcount\c@pp
       \newcount\c@mm
\def\insertsectionnavigation#1{%
  \hbox to #1{%
    \vbox{{\usebeamerfont{section in head/foot}\usebeamercolor[fg]{section in head/foot}%
     \vskip0.5625ex%
    \def\slideentry##1##2##3##4##5##6{}%
     \def\sectionentry##1##2##3##4##5{%
       \ifnum##5=\c@part%
       \def\insertsectionhead{##2}%
       \def\insertsectionheadnumber{##1}%
       \def\insertpartheadnumber{##5}%
       \c@p=\c@section%
       \c@m=\c@section%
      \c@pp=\c@section%
       \c@mm=\c@section%
       \advance\c@m by -1 %
       \advance\c@p by 1 %
       \advance\c@mm by -2 %
       \advance\c@pp by 2 %
       %
       \ifnum \c@section=1
                    \ifnum\c@section=##1%
               \setbox\beamer@tempbox=\hbox{%
              \hyperlink{Navigation##3}{\hbox to #1{%
             {\hskip0.3cm\usebeamertemplate{section in head/foot}\hskip0.3cm}}}}%
             \else%
                 \ifnum##1=\c@p%
                 \setbox\beamer@tempbox=\hbox{%
                  \hyperlink{Navigation##3}{\hbox to #1{%
                 {\hskip0.3cm\usebeamertemplate{section in head/foot shaded}\hskip0.3cm}}}}
                 %
                \else%
                 \ifnum##1=\c@pp%
                 \setbox\beamer@tempbox=\hbox{%
                  \hyperlink{Navigation##3}{\hbox to #1{%
               {\hskip0.3cm\usebeamertemplate{section in head/foot shaded}\hskip0.3cm}}}}%
               %
               \else%
               %
               \fi%
               \fi%
               %
            \fi%%
         \else%
  \ifnum \c@section=\beamer@sectionmax
               \ifnum\c@section=##1%
               \setbox\beamer@tempbox=\hbox{%
              \hyperlink{Navigation##3}{\hbox to #1{%
             {\hskip0.3cm\usebeamertemplate{section in head/foot}\hskip0.3cm}}}}%
             \else%
                 \ifnum##1=\c@m%
                 \setbox\beamer@tempbox=\hbox{%
                  \hyperlink{Navigation##3}{\hbox to #1{%
                 {\hskip0.3cm\usebeamertemplate{section in head/foot shaded}\hskip0.3cm}}}}
                 %
                \else%
                 \ifnum##1=\c@mm%
                 \setbox\beamer@tempbox=\hbox{%
                  \hyperlink{Navigation##3}{\hbox to #1{%
               {\hskip0.3cm\usebeamertemplate{section in head/foot shaded}\hskip0.3cm}}}}%
               %
               \else%
               %
               \fi%
               \fi%
               %
            \fi%%
       \else%
             \ifnum\c@section=##1%
               \setbox\beamer@tempbox=\hbox{%
              \hyperlink{Navigation##3}{\hbox to #1{%
             {\hskip0.3cm\usebeamertemplate{section in head/foot}\hskip0.3cm}}}}%
             \else%
                 \ifnum##1=\c@m%
                 \setbox\beamer@tempbox=\hbox{%
                  \hyperlink{Navigation##3}{\hbox to #1{%
                 {\hskip0.3cm\usebeamertemplate{section in head/foot shaded}\hskip0.3cm}}}}
                 %
                \else%
                 \ifnum##1=\c@p%
                 \setbox\beamer@tempbox=\hbox{%
                  \hyperlink{Navigation##3}{\hbox to #1{%
               {\hskip0.3cm\usebeamertemplate{section in head/foot shaded}\hskip0.3cm}}}}%
               %
               \else%
               %
               \fi%
               \fi%
               %
            \fi%%
            %
            \fi
            \fi
            %

      \ht\beamer@tempbox=1.6875ex%
       \dp\beamer@tempbox=0.75ex%
       \box\beamer@tempbox\fi}%
     \dohead\vskip0.5625ex}}\hfil}}


\setbeamertemplate{headline}%{split theme} % full manual adjustment
{%
  \leavevmode%
  \@tempdimb=3em%
  \ifdim\@tempdimb>0pt%
    \advance\@tempdimb by 1.825ex%
    \begin{beamercolorbox}[wd=.5\paperwidth,ht=\@tempdimb]{section in head/foot}%
      \vbox to\@tempdimb{\vfil\insertsectionnavigation{.5\paperwidth}\vfil}%
    \end{beamercolorbox}%
    \begin{beamercolorbox}[wd=.5\paperwidth,ht=\@tempdimb]{subsection in head/foot}%
      \vbox to\@tempdimb{\vfil\insertsubsectionnavigation{.5\paperwidth}\vfil}%
    \end{beamercolorbox}%
  \fi%
}



\makeatother


\begin{document}

  \frame{\titlepage}

  \section*{Überblick}
  \frame{
    \frametitle{Überblick}
    \tableofcontents
  }

  \section{Ursprung der parallelen Programmierung}

  \frame{
    \frametitle{Ursprung der parallelen Programmierung}
    \begin{itemize}
      \item Sequentielle Programme wurden immer leistungshungriger
      \item Chiphersteller reagieren auf die Anforderungen der Programmierer
      \item Erhöhen der Anzahl an Transistoren
      \item Erhöhen der Taktrate
    \end{itemize}
    Problem? $\rightarrow$ Abwärme \\
    Lösung? $\rightarrow$ Mehrprozessorsysteme
  }

  \frame{
    \frametitle{Ursprung der parallelen Programmierung}
    \begin{itemize}
      \item Verwenden eines Mehrprozessorsystems heißt nicht automatisch
      Leistungszuwachs
      \item Programmierer reagieren auf die Anforderungen der Chiphersteller
      \item Parallele Programme sind fehleranfällig
      \item Debuggen paralleler Programme ist schwer
    \end{itemize}
  }

  \section{Probleme beim Debuggen paralleler Programme}
  \frame{
    \frametitle{Probleme beim Debuggen paralleler Programme}
    \begin{itemize}
      \item Fehler können nur schwer reproduziert werden
      \item Je nach Grad der Parallelität ist das Debuggen mit enormen
      Zeitaufwand verbunden
      \item Durch das Verwenden eines Debuggers leidet die
      Ausführungsgeschwindigkeit des Programms und es können neue Fehler
      auftreten oder vorhandene Fehler verschwinden
    \end{itemize}
  }

  \section{Arten von Fehlern}
  \frame{
    \frametitle{Arten von Fehlern}
    \begin{itemize}
      \item<1->{\only<2>{\color{blue}} Race Conditions}
      \only<2>{\\2+ Prozesse/Threads haben unsynchronisierten Zugriffen auf eine
      Variable}
      \item<1->{\only<3>{\color{blue}} Deadlock}
      \only<3>{\\2+ Prozesse/Threads blockieren sich gegenseitig}
      \item<1->{\only<4>{\color{blue}} Livelock}
      \only<4>{\\2+ Prozesse/Threads verhindern die weitere Ausführung des
      Programms}
      \item<1->{\only<5>{\color{blue}} Starvation}
      \only<5>{\\Ein Prozess/Thread verhungert, falls dieser nicht mehr zum Zug
      kommt}
      \item<1->{\only<6>{\color{blue}} Heisenbug}
      \only<6>{\\Fehler, die durch das Debuggen verschwinden oder hervorgerufen
      werden}
    \end{itemize}
  }

  \section{Debugging Strategien}
  \frame{
    \frametitle{Debugging Strategien}
    \begin{itemize}
      \item<1->{\only<2>{\color{blue}} Single-Step Debugging}
      \only<2>{\\Schrittweises Debuggen, Zeile für Zeile}
      \item<1->{\only<3>{\color{blue}} Breakpointing}
      \only<3>{\\Stoppen des Programms an einem gewissen Punkt}
      \item<1->{\only<4>{\color{blue}} Checkpointing}
      \only<4>{\\Einbauen von Sicherungspunkten}
      \item<1->{\only<5>{\color{blue}} Cyclic Debugging}
      \only<5>{\\Wiederholtes Ausführen des Programms zum Finden der
      Fehlerursache}
      \item<1->{\only<6>{\color{blue}} Reverse Debugging}
      \only<6>{\\Schrittweises Rückwartsdebuggen}
      \item<1->{\only<7>{\color{blue}} Record/Replay}
      \only<7>{\\Aufzeichnen und Abspielen eines Programms unter denselben
      Vorraussetzungen}
      \item<1->{\only<8>{\color{blue}} Program Slicing}
      \only<8>{\\Abstrahieren eines Programmteils um die Fehlersuche
      einzuschränken}
    \end{itemize}
  }

  \section*{Fragen?}
  \frame{
    \begin{center}
      \Huge Fragen?
    \end{center}
  }

\end{document}
